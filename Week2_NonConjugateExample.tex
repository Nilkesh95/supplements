\documentclass{article}
\usepackage{indentfirst,latexsym,bm}
\usepackage{graphics}
\usepackage{color}
%\usepackage{CJK}
\usepackage{amsmath}
\usepackage{amssymb}
\usepackage{dsfont}
\usepackage{mathrsfs} %For \mathscr
\usepackage[latin1]{inputenc}
\usepackage{tikz}
\usetikzlibrary{trees}
\usepackage{listings}
\usepackage{csquotes}
\usepackage[top=0.8in, bottom=.8in, left=0.5in, right=0.5in]{geometry}
\setlength{\parindent}{1em}

\def\dsst{\displaystyle}
\pagenumbering{arabic}
\begin{document}
%\pagestyle{empty}
\title{Week 2 Non Conjugate Example (Optional)}
\author{Duke University}
\date{}
\maketitle 

%\section*{}



In this file, we will solve our usual coin toss problem, but to use a prior distribution that does not pair with the Binomial distribution to for conjugacy. Since there is no conjugacy we can apply, we will need to carry out integral calculations.

\subsection*{Problem}
\begin{displayquote}
	You have a coin with unknown chance $p$ of getting heads. Suppose you believe the chance of getting heads is more likely that the chance of getting tails. So you place a prior distribution $\dsst \pi(p) = \frac{2}{3}(1+p),\ 0\leq p \leq 1$. You toss the coin 3 times and get 1 head. What would the posterior distribution of $p$ be based on the results of the 3 coin tosses? 
\end{displayquote}

\subsection*{Prior}

The prior distribution we use here is
$$ \pi(p) = \frac{2}{3}(1+p),\qquad 0\leq p \leq 1. $$

This is a function I randomly pick. The graph of the function can be shown below, with the red dash-line indicating the mean of $p$. This is an increasing straight line, concentrating more mass in the region when $p$ is closer to 1 than $p$ is to 0. So this reflects our belief that there is higher chance to get heads than tails.
\begin{figure}[tbph]
	\centering
	\includegraphics[width=0.4\linewidth]{C:/Files/Jobs/DukeCourseraMentor/Week2/NonConjugateExample1}
	%\caption{}
	%\label{fig:nonconjugateexample1}
\end{figure}


We can check that this function satisfies the properties of probability distributions:
\begin{itemize}
	\item $\pi(p)\geq 0$ for any $p$ in the defined range ($0\leq p \leq 1$),
	
	\item $\dsst \int_0^1 \pi(p)\, dp = \int_0^1 \frac{2}{3}(1+p)\, dp = 1$.
\end{itemize}

So we can use this function as our prior distribution of $p$.

\subsection*{Prior Mean}

Before we carry out the Bayes' Rule, let us first calculate the mean of this prior distribution and make comparison with the posterior later. \\

The prior mean reflects the average of $p$, the chance of getting heads, under our belief before we have any data. The following calculation tells us the prior mean is:
$$ \text{prior mean of $p$ }=\int_0^1 p\pi(p)\, dp = \int_0^1 p\left[\frac{2}{3}(1+p)\right]\, dp = \frac{5}{9}\approx 0.5555556 > 0.5.$$

\subsection*{Use Bayes' Rule to Update}

The prior distribution $\dsst \pi(p) = \frac{2}{3}(1+p)$ does not belong to the Beta distribution family since it does not carry a form of $p^{\text{some power}}(1-p)^{\text{another power}}$. To update and obtain the posterior distribution of $p$ after observing 1 head in 3 tosses, we have to use the formula of Bayes' Rule:

\begin{align*}
\pi^*(p~|~\text{1 head in 3 tosses}) = & \frac{\mathbb{P}(\text{1 head in 3 tosses}~|~p)\times \pi(p)}{\dsst \int_{0}^{1}\mathbb{P}(\text{1 head in 3 tosses}~|~p)\times \pi(p)\, dp} = \frac{\dsst \left[\binom{3}{1}p^1(1-p)^2\right]\left[\frac{2}{3}(1+p)\right]}{\dsst \int_0^1  \left[\binom{3}{1}p^1(1-p)^2\right]\left[\frac{2}{3}(1+p)\right]\, dp} \\
& \\
= & \frac{2(p-p^2-p^3+p^4)}{\dsst \int_0^1 2(p-p^2-p^3+p^4)\, dp} = \frac{60}{7}(p-p^2-p^3+p^4).
\end{align*}

So the posterior distribution of $p$ in this problem has become 
$$ \pi^*(p~|~\text{data}) = \frac{60}{7}(p-p^2-p^3+p^4),\qquad 0\leq p\leq 1. $$
with its graph shown below, with the dash-line indicating the posterior mean of $p$.
\begin{figure}[tbph]
	\centering
	\includegraphics[width=0.4\linewidth]{C:/Files/Jobs/DukeCourseraMentor/Week2/NonConjugateExample2}
	%\caption{}
	\label{fig:nonconjugateexample2}
\end{figure}

\subsection*{Posterior Mean}

We can see that the posterior distribution has dramatically move its weight away from 1, after getting the data. We can calculate the mean of the posterior distribution and compare:
$$ \text{posterior mean of $p$ }=\int_0^1 p\pi^*(p~|~\text{data})\, dp = \int_0^1 p\left[\frac{60}{7}(p-p^2-p^3+p^4)\right]\, dp = \frac{3}{7} \approx 0.4285714 < 0.5.$$ 

The data that we only have 1 head out of 3 tosses have shifted our belief of $p$ from getting closer to 1 to getting closer to 0.







\end{document}